\begin{section}

\section{Introduction}
At \acf{CERN} in Switzerland there are being conducted experiment on fundamental structure of the universe. This is done by accelerating particles up to a energy of 7 $\tera \electronvolt$ per proton, and then crash with a other particles with same energy. This experiment is done by connecting several accelerators with higher and higher energies together, the largest one is called \acf{LHC}, and is the largest particle accelerator ever built, installed in a 27 $\kilo \meter$ long tunnel.
To detect what is happening to the particles when crashed, there has been build several detectors that is placed in the tunnel. One of these is called the \acf{ALICE} detector. The \ac{ALICE} detector is using electronics to measure and detect what is happening under a crash.

The Physics and research group at the \acf{UiB} has been working with \ac{CERN} on the \ac{ALICE} project since it started.
One of the main boards used in the \ac{ALICE} detector is the \acf{RCU}. Now there has been decided that a new \ac{RCU} board shall be made, that is called \ac{RCU2}.
Everything that are going to be used at \ac{CERN} has to be made sure that it can survive in the radiation level that can occur there. Therefore every \acf{IC} planned to be used for the design of the \ac{RCU2} board has to be tested for radiation to be sure that it won't fail when it is installed in the \ac{ALICE} detector.

\subsection{How to test}
\label{how_to_test}
The radiation in the \ac{LHC} is dominated by high energetic neutrons and protons, mostly neutrons with a estimated fluence of $(0,6-1,1)\times 10^{11}$ $neutrons/cm^2$. Therefore it would be preferable to test our electronics with a neutron beam, but since there are few labs who can produce a neutron beam compared to proton beam most of the electronics is only tested at \ac{OCL} with a proton beam.
There has been done experiment that compares SEU induced by neutrons and protons \cite{GranlundOlsson}, and the result shows that it is possible to use a proton beam instead of proton beam with small deviations. By comparing a Proton beam with a neutron beam of 21MeV we see that we get 10-25\% less SEU cross section for a proton beam compared to a neutron beam.
If we increase the energy to 88MeV then we get close to none deviations. 

The tests that are done through this thesis are so called dose-tests. That is irradiation up to a level where an error can clearly be seen or when a high enough dose has been reached without detecting errors.
Current consumption and the outputs of the \ac{IC} are monitored through the whole irradiation process.
The dose that we could expect at \ac{CERN} for a 10 year period in the ALICE detector is estimated to be approximately 0.6 kRad from Pb-Pb collisions that will be run 1 month a year and a little higher for p-p collisions that will be run 10 months a year \cite{georgios} and \cite{roed}.
Therefore we could expect a dose of 1-2 $\kilo$Rad during the time it will be used at CERN.
If a \ac{IC} survives more than 5 times of what we would expect at \ac{CERN}, we could say that it pass the test, that means if it survives more than 5-10 kRad, the device is approved to used in the \ac{RCU2} design.

\subsection{About this work}
When I started working with my thesis in the autumn of 2013 the schematic layout for the \ac{RCU2} was basically finished, and most of the component was decided, but not everyone had been tested.
So what I have been working on in my thesis are thinking and planing how to test the different \ac{IC}.
In most of the cases I designed a simple test \ac{PCB} which I connected to \acf{DAQ} board from \ac{NI}. By the use of this \ac{DAQ} the functionality of the \ac{IC} was tested and current consumption was monitored.
For the more advanced \ac{IC} like the \ac{SF2}\acf{SOC}\ac{FPGA}, I used a starter-kit when designing the test. To measure the low current going into the \ac{SF2} chip I also made a current measurement \ac{PCB} for that purpose.
I started my thesis work with making test boards for 8 different \ac{IC}s. These consist of; power regulators, bus transceivers, limiting amplifier, multiplexer/demultiplexer and buffer.
For every each of the different test board that was made I also made labVIEW programs, to control and monitor the tests.
After these test board was made working and had been tested, I started working on designing test for the \ac{SF2} \ac{SOC} \ac{FPGA}.
The tests was made on a \ac{SF2}-starter kit, but with the \ac{RCU2} in mind, so that when the hardware for the \ac{RCU} comes, it will be easy to implement the test on that.
I also made two more test board for two IC that was added to the design at a late stage. That was a comparator and a Current Shunt Monitor.

\end{section}
