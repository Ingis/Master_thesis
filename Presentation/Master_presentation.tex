\documentclass[]{beamer}

%
% Use this to create handout pages
%
%\documentclass[handout]{beamer}
%\usepackage{pgf,pgfpages}
%\pgfpagesuselayout{2 on 1}[a4paper, portrait, border shrink=0.5in]

\usepackage{graphicx}
\usepackage[utf8]{inputenc}
\usepackage[T1]{fontenc}
\usepackage{times}
% \usepackage{babel}

\mode<presentation>
{
  \usetheme{ift}
  \setbeamercovered{transparent}
  \setbeamertemplate{items}[square]
}

\usefonttheme[onlymath]{serif}
\setbeamerfont{frametitle}{size=\LARGE,series=\bfseries}
\setbeamerfont{title}{size=\LARGE,series=\bfseries}

\beamertemplatenavigationsymbolsempty


%\includeonlyframes{current}

\title{Beamer Template UiB}
\author{Mr. Smith}
\date{\today}
\institute{Institutt for Fysikk og Teknologi, Universitetet i Bergen}

\begin{document}


\setbeamertemplate{background}
 {\includegraphics[width=1.0005\paperwidth,height=\paperheight]{frontpage_bg}}
\setbeamertemplate{footline}[default]

\begin{frame}
  \titlepage
  \vspace{5cm}
\end{frame}

%
% Set the background for the rest of the slides.
% Insert infoline at the end
%
\setbeamertemplate{background}
 {\includegraphics[width=1.0005\paperwidth,height=\paperheight]{slide_bg}}
\setbeamertemplate{footline}[ifttheme]

%--------------------------------------------------------------------------------------------------
% SECTION: First section
%--------------------------------------------------------------------------------------------------
\section{First section}

%
% Introduction
%
\begin{frame}
	\frametitle{Introslide}

	\begin{columns}
		\column{.5\textwidth}
			\begin{itemize}
				\item<2-> SubItem
			\end{itemize}
		\column{.5\textwidth}
	\end{columns}
\end{frame}



%--------------------------------------------------------------------------------------------------
% SECTION: Another section
%--------------------------------------------------------------------------------------------------
\section{Anoter section}

%
% The Helium atom
%
\begin{frame}
	\frametitle{More slides!}
	\begin{columns}
		\column{.5\textwidth}
			\begin{itemize}
				\item<1-> one
				\item<2-> two
					\begin{itemize}
						\item<2-> sub-two
					\end{itemize}
				\item<3-> three
				\item<4-> four
			\end{itemize}
				
		\column{.5\textwidth}
			\includegraphics[width=\columnwidth]{images/atom-newton}
	\end{columns}
\end{frame}


\end{document}